% ----------------------------------------------------------------------
% Poster for Bryanna 2022 -- SWPA, Pathways, and more! 
% ---------------------------------------------------------------------

\documentclass[paperheight=36in,paperwidth=48in,landscape]{baposter}
\usepackage{amsmath}
\usepackage{caption}
\captionsetup[figure]{labelformat=empty}
\usepackage{relsize}  % For \smaller
\usepackage{url}      % For \url
\usepackage{tikz}
\usepackage{xcolor}
\usetikzlibrary{shapes.misc}

%\usepackage{sansmathfonts}
\usepackage[T1]{fontenc}
\renewcommand*\familydefault{\sfdefault}

%%% Global Settings %%%%%%%%%%%%%%%%%%%%%%%%%%%%%%%%%%%%%%%%%%%%%%%%%%%%

\graphicspath{{figures/}}  % Root directory of the pictures 
% \tracingstats=2  % Enabled LaTeX logging with conditionals

%%% Color Definitions %%%%%%%%%%%%%%%%%%%%%%%%%%%%%%%%%%%%%%%%%%%%%%%%%%

\definecolor{bordercol}{RGB}{40,40,40}
\definecolor{tarletonPurple}{RGB}{79,45,127}
\definecolor{headerfontcol}{RGB}{0,0,0}
\definecolor{boxcolor}{RGB}{255,255,255}
\definecolor{backgroundcolor}{RGB}{229,224,236}
% background color is computed by extending each RGB component of tarletonPurple
% toward white (255,255,255) in a proportional manner. This specific choice
% is 85% of the way toward white.  This way, it is not too dark, but still
% retains some of the visual appeal of Tarleton purple

%%%%%%%%%%%%%%%%%%%%%%%%%%%%%%%%%%%%%%%%%%%%%%%%%%%%%%%%%%%%%%%%%%%%%%%%
%%% Utility functions %%%%%%%%%%%%%%%%%%%%%%%%%%%%%%%%%%%%%%%%%%%%%%%%%%

%%% Save space in lists. Use this after the opening of the list
\newcommand{\compresslist}{
	\setlength{\itemsep}{0.2pt}
	\setlength{\parskip}{0pt}
	\setlength{\parsep}{0pt}
}


%%%%%%%%%%%%%%%%%%%%%%%%%%%%%%%%%%%%%%%%%%%%%%%%%%%%%%%%%%%%%%%%%%%%%%%%%
%%% Document Start %%%%%%%%%%%%%%%%%%%%%%%%%%%%%%%%%%%%%%%%%%%%%%%%%%%%%%
%%%%%%%%%%%%%%%%%%%%%%%%%%%%%%%%%%%%%%%%%%%%%%%%%%%%%%%%%%%%%%%%%%%%%%%%%

\begin{document}

%%% General Poster Settings %%%%%%%%%%%%%%%%%%%%%%%%%%%%%%%%%%%%%%%%%%%%
%%%%%% Eye Catcher, Title, Authors and University Images %%%%%%%%%%%%%%%
\begin{poster}{
        headerheight=0.1\textheight,
	grid=false,
        columns=3,
    	eyecatcher=true, 
	borderColor=bordercol,
	headerColorOne=tarletonPurple,
        headershade=plain,
	headerFontColor=boxcolor,
	% Only simple background color used, no shading, so boxColorTwo
        % isn't necessary
	boxColorOne=boxcolor,
	headershape=smallrounded,
	headerfont=\Large\sf\bf,
	textborder=roundedsmall,
	background=plain,
        bgColorOne=backgroundcolor,
        headerborder=closed,
        boxshade=plain
}
%%% Eye Catcher %%%%%%%%%%%%%%%%%%%%%%%%%%%%%%%%%%%%%%%%%%%%%%%%%%%%%%%%%
{
	
}
%%% Title %%%%%%%%%%%%%%%%%%%%%%%%%%%%%%%%%%%%%%%%%%%%%%%%%%%%%%%%%%%%%%%
{\LARGE\bf
  Classical Maximum Likelihood Estimation in Shifted-Wald Models of Response Times
}
%%% Authors %%%%%%%%%%%%%%%%%%%%%%%%%%%%%%%%%%%%%%%%%%%%%%%%%%%%%%%%%%%%%%%%%%%
{
  { Bryanna L. Scheuler \& Thomas J. Faulkenberry}\\
  { Department of Psychological Sciences, Tarleton State University}\\
}
%%% Logo %%%%%%%%%%%%%%%%%%%%%%%%%%%%%%%%%%%%%%%%%%%%%%%%%%%%%%%%%%%%%%%%%%%%%%
{
 \includegraphics[height=0.8\headerheight]{figures/logo.jpg} 
}
%%%%%%%%%%%%%%%%%%%%%%%%%%%%%%%%%%%%%%%%%%%%%

\headerbox{Background}{name=background,span=1,column=0,row=0}{ 

Response times are a crucial component of understanding and measuring cognitive processes. There are many probability distributions that can be used to describe distributions of response times. Thus, given some observed data, we use various techniques to estimate the {\color{blue} population parameters} of the distribution that could have potentially generated the observed sample. But, {\color{blue} how do we know that the parameters we are estimating from a response time distribution are representative of the true population parameters?} To assess the validity of our estimation technique, we must conduct {\color{blue} a parameter recovery study}.

  \begin{center}
    \includegraphics[width=0.4\textwidth]{figures/ParameterRecovery.jpg}
  \end{center}

This study focused on {\color{red}classical maximum likelihood estimation (CMLE)}, a method that works by maximizing the likelihood function that results when we observe some data (Myung, 2003). While one can use calculus to maximize some likelihood functions, the more common technique is to use computer methods to {\color{red}minimize} the negative log-likelihood function. These techniques are conceptually easy to describe, but there are instances where the algorithm fails (e.g., stuck in a local minimum, etc.). Thus, it is essential to perform a parameter recovery study to gauge whether our CMLE algorithm can actually return the correct parameter values to us.

We tested CMLE on a specific model of response times known as the {\color{blue} shifted-Wald model}. The shifted-Wald model is composed of three parameters: shift, drift rate, and response threshold.


  \begin{center}
    \includegraphics[width=0.6\textwidth]{figures/sWaldModel.jpg}
  \end{center}
}
%%%%%%%%%%%%%%%%%%%%%%%%%%%%%%%%%%%%%%%%%%%

\headerbox{Method}{name=method, span=1, below=background, above=bottom}{

For this study, we simulated data using shifted-Wald parameter targets reported by Faulkenberry et al. (2018). This was composed of four main steps:

  \begin{itemize} \compresslist
  \item Generate 'artificial' people from parent population
  \item Generate response times for artifical people
  \item Fit shifted-Wald model to RT distribution with CMLE
  \item Compare estimates to original target parameters
  \end{itemize}

This process was applied to 5 sub-experiments in a design used by Farrell and Ludwig (2008). These sub-experiments had 5, 20, or 80 participants, with 20, 80, or 500 trials per particpant.

}

%%%%%%%%%%%%%%%%%%%%%%%%%%%%%%%%%%%%%%%%%%%%%%%

\headerbox{Experiments}{name=experiments, span=1,column=1, row= 1, above=bottom}{

Experiment 1: Participants = 20, Trials = 20
 \begin{center}
    \includegraphics[width=0.57\textwidth]{figures/exp1Cor.jpg}
  \end{center}
 \begin{center}
    \includegraphics[width=0.45\textwidth]{figures/exp1Results.jpg}
  \end{center}

Experiment 2: Participants = 20, Trials = 80
 \begin{center}
    \includegraphics[width=0.57\textwidth]{figures/exp2Cor.jpg}
  \end{center}
 \begin{center}
    \includegraphics[width=0.45\textwidth]{figures/exp2Results.jpg}
  \end{center}

Experiment 3: Participants = 20, Trials = 500
 \begin{center}
    \includegraphics[width=0.57\textwidth]{figures/exp3Cor.jpg}
  \end{center}
 \begin{center}
    \includegraphics[width=0.45\textwidth]{figures/exp3Results.jpg}
  \end{center}

Experiment 4: Participants = 5, Trials = 500
 \begin{center}
    \includegraphics[width=0.57\textwidth]{figures/exp4Cor.jpg}
  \end{center}
 \begin{center}
    \includegraphics[width=0.45\textwidth]{figures/exp4Results.jpg}
  \end{center}

Experiment 5: Participants = 80, Trials = 20
 \begin{center}
    \includegraphics[width=0.57\textwidth]{figures/exp5Cor.jpg}
  \end{center}
 \begin{center}
    \includegraphics[width=0.45\textwidth]{figures/exp5Results.jpg}
  \end{center}
}
%%%%%%%%%%%%%%%%%%%%%%%%%%%%%%%%%%%%%%%%%%%%%%%%%%%%%

\headerbox{Recovering drift rate?}{name=driftrate, span=1, column=2}{
\begin{minipage}{0.5\textwidth}
The target mean for drift rate was {\color{blue} 3.91}. The mean estimated drift rates were:
\\Experiment 1 = 4.49 (0.60)
\\Experiment 2 = 3.97 (0.23)
\\Experiment 3 = 3.91 (0.17)
\\Experiment 4 = 3.92 (0.32)
\\Experiment 5 = 4.52 (0.31)
\end{minipage}%
\begin{minipage}{0.5\textwidth}
    \includegraphics[width=0.95\textwidth]{figures/driftrate.jpg}
\end{minipage}
}
%%%%%%%%%%%%%%%%%%%

\headerbox{Recovering response threshold?}{name=responsethreshold, span=1, column=2, below=driftrate}{
\begin{minipage}{0.5\textwidth}
The target mean for response threshold was {\color{blue} 0.92}. The mean estimated response thresholds were:
\\Experiment 1 = 1.88 (0.98)
\\Experiment 2 = 0.95 (0.08)
\\Experiment 3 = 0.92 (0.04)
\\Experiment 4 = 0.92 (0.09)
\\Experiment 5 = 1.89 (0.45)
\end{minipage}%
\begin{minipage}{0.5\textwidth}
    \includegraphics[width=0.95\textwidth]{figures/responsethreshold.jpg}
\end{minipage}
}

%%%%%%%%%%%%%%%%%%%%

\headerbox{Recovering shift?}{name=shift, span=1, column=2, below=responsethreshold}{
\begin{minipage}{0.5\textwidth}
The target mean for shift was {\color{blue} 0.32}. The mean estimated shifts were:
\\Experiment 1 = 0.27 (0.06)
\\Experiment 2 = 0.32 (0.01)
\\Experiment 3 = 0.32 (0.01)
\\Experiment 4 = 0.32 (0.02)
\\Experiment 5 = 0.27 (0.03)
\end{minipage} %
\begin{minipage}{0.5\textwidth}
    \includegraphics[width=0.95\textwidth]{figures/shift.jpg}
\end{minipage}
}

%%%%%%%%%%%%%%%%%%%%%

\headerbox{Discussion}{name=discussion, span=1, column=2, below=shift, above=bottom}{
\begin{minipage}{0.5\textwidth}\small
There are still many questions left to be answered:
  \begin{itemize} \compresslist
  \item How do other methods compare to CMLE when estimating shifted-Wald parameters?
  \item Is there an optimum trial size when using CMLE for these models?
  \item How can we use this information about participant versus trial size to make informed decisions about sampling plans?
  \end{itemize}
\end{minipage} %
\begin{minipage}{0.5\textwidth}
  \centering
    \includegraphics[width=0.8\textwidth]{figures/qrcode.jpg}
 \end{minipage}

}

\end{poster}
\end{document}
